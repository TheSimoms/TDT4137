\documentclass{article}


% Packages
\usepackage[utf8]{inputenc} % For Norwegian letters
\usepackage{tabulary} % For nice tables

% Config
\renewcommand\thesection{\alph{section}}


\begin{document}

% Title
\title{\textbf{Exercise 1} \\ TDT4137}
\author{Simon Borøy-Johnsen \\ MTDT}
\date{\today}
\maketitle

% Content
% Task a)
\section{}
Assuming eye movement is not necessary, as one normally will detect that the brake lights are lit without looking directly at them. \\

\begin{tabulary}{\textwidth}{|L|L|}
	\hline
	\textbf{Event} & \textbf{Time passed} \\\hline
	Break light turns red & $\tau=0.00$ \\\hline
	Perceptual processor fetches the image into VIS and WM & $\tau=\tau_p$ \\\hline
	Cognitive processor generates the motor command to press the brakes & $\tau=\tau_p+\tau_c$ \\\hline
	Motor processor executes the command & $\tau=\tau_p+\tau_c+\tau_m$ \\\hline
	DONE & $\tau=\tau_p+\tau_c+\tau_m$\newline$\tau=100ms+70.0ms+70.0ms$\newline$\tau=240ms$ \\\hline
\end{tabulary}

% Task b)
\section{}
Assuming eye movement is not necessary. \\

\begin{tabulary}{\textwidth}{|L|L|}
	\hline
	\textbf{Event} & \textbf{Time passed} \\\hline
	Flag appears on monitor & $\tau=0.00$ \\\hline
	Perceptual processor fetches the image into VIS and WM & $\tau=\tau_p$ \\\hline
	Fetching the semantic name from LTM to WM & $\tau=\tau_p+\tau_c$ \\\hline
	Fetching that the flag is Scandinavian from LTM & $\tau=\tau_p+\tau_c+\tau_c$ \\\hline
	DONE & $\tau=\tau_p+\tau_c+\tau_c$\newline$\tau=100ms+70.0ms+70.0ms$\newline$\tau=240ms$ \\\hline
\end{tabulary}

% Task c)
\section{}
The Index of Difficulty describes the difficulty of performing a movement task accurately. According to Fitts, the Index of Difficulty is $log_2(D/S+0.5)$, where D is the distance to the target and S is the size of the target.

Using Shannon's version of Fitts' law to calculate the duration of moving the cursor to the menu bar, we get the following results: \\
\begin{tabulary}{\textwidth}{L L L}
	Windows: & $50.0+150*log_2(80.0/5.00+1.00)=663ms$ &  \\
	Machintosh: & $50.0+150*log_2(80.0/50.0+1.00)=257ms$ &  \\
\end{tabulary}

% Task d)
\section{}
To create an illusion of continuity in time, there has be be more images than perceptual cycles per time unit. Frame rate (fr) has to be bigger than $1/\tau_p$. This way the stream of perceptual images 'melts' together to one continuous impression.

$fr>1.00/\tau_p=1.00/100ms=10.0/s$ \\
The frame rate has to be larger than 10 frames per seconds.

\end{document}